\section{Introduction}
CloudMan is a high level resource management tool which provides a central place to configure resources allocations in a computer center. Users who authenticate to CloudMan have privileges which depend on the role they have been assigned by somebody else. This way CloudMan allows for a delegation of configuration details to user communities. As an example, a CloudMan super user can create a new user group {\it A} and assign computing resources in specific parts of the computer center to this group of people. As an example, group {\it A} can correspond to the list of computing coordinators of a big LHC experiment. 
The people belonging to group {\it A} can decide themselves how they want to make use of those resources. Let's assume that the computer center offers two projects {\it X} and {\it Y} which are interesting for the user group {\it A}. Examples of those projects can be a classical batch system, providing resources which are available from the Grid, and a virtual machine self-service. So they decide to use 90\% of the resource for classical batch while the rest should be used for a self-service. 
As the people from group {\it A} are usually busy people, they can further delegate the  management of the resources within the project. They could found a batch group for the experiment which can give the bulk of the batch resources to the Higgs research group, and the rest to some students. 

The CloudMan front end takes care of the hierachies and privileges. The result of the allocation is exported in a machine readable format, and can then be used by scripts to configure the services behind the projects. 
