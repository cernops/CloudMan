At the time of writing this document, AFS resources at CERN are managed like this:

\begin{itemize}
\item the AFS service provide so called "project spaces"
\item in 1st approximation project spaces are defined by a project name, e.g. "atlas", a path, e.g. "/afs/cern.ch/atlas", and a quota, "e.g. 10TB"
\item the management of the project spaces is outsourced to so-called project admins
\item being a project admin basically means being member of the AFS group "\_projectname\_", e.g. "\_atlas\_"
\item project admin's task include
\begin{itemize}
    \item creating/deleting different types of volumes
    \item managing ACLs
    \item managing readonly volumes
\end{itemize}
\item tool to manage all this is afs\_admin (running it as an admin of a project w/o any option will list all the subcommands)    
\end{itemize}

The AFS service managers see a possible benefit in managing at least parts of the AFS resource~\cite{arne}. In particular:

\begin{itemize}
\item quicker overview of the managed AFS project space
\item easier to execute specific commands (e.g. if CloudMan "knows" to required parameters and can do some sanity checking)
\end{itemize}

Requirements for a potential web-interface include:

\begin{itemize}
\item CloudMan cannot be the primary source of any information, so it must be able to retrieve information from somewhere else (like parsing a file or running a command)
\item CloudMan must be able to somehow trigger a authenticated command (afs\_admin) on behalf of the user
\item CloudMan must be able to give feedback to the user if a command succeeded or not
\item extending/changing afs\_admin should not require changes in CloudMan and it should support the full command set
\item AFS project administration is based on AFS groups, not e-groups, so CloudMan needs to either grab that info from the AFS group or establish a sync from AFS into e-groups 
\end{itemize}
